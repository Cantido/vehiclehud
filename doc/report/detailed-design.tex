\chapter{Detailed Design}

\section{PSU}

\section{Main Board}

\section{Raspberry Pi}

A Raspberry Pi computer was chosen to be high-level platform for this system
because of its popularity and ease of use. The Raspberry Pi is connected to the
main board via USB, and delegates the tasks of gathering and displaying data
to the main board.

\subsection{Operating System}
The operating system chosen for the Raspberry Pi was Rasbian, which is the
most used and well-known operating system for the Raspberry Pi. Rasbian was
chosen for these reasons as well: we wanted to focus on designing the overall
system, not on wrestling with an obscure operating system. We prioritized ease
of development.

\section{Software}
The software for the system is written in Python 2.7.5. Python was chosen
because it is the language of choice for the Raspberry Pi, and allows us to
quickly and 
exibly develop the high-level system software. The version was
chosen for compatibility with the PySerial library, which the software uses to
communicate over USB.

Source code for the system (as well as source code for this report) can be
found at \url{https://github.com/Cantido/vehiclehud}.
