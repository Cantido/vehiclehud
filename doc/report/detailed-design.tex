\chapter{Detailed Design}

% This section will form the bulk of the report. Here you should include
% design decisions and tradeoffs as well as any detailed technical drawings
% such as circuit diagrams, flow charts, etc.  If these are particularly large
% they may be placed in an appendix, but should be referenced in this section.

% It is here that you really get into the details of why your project is
% designed the way it is.  Tradeoffs are made in a number of areas and a good
% way to organize this section is to figure out what the most important
% tradeoffs are and explain each of them with a few paragraphs.

% This is a section that will evolve as the project nears completion, but its
% writing could be started at the beginning of the semester.  I'll be happy to
% give feedback on whatever you are able to produce for this section during the
% semester, while recognizing that some elements will be subject to change or
% impossible to write up until the final project is completed.

\section{PSU}
Due to the volatile nature of automotive electrical systems, a very stable and 
resilient power supply is required for this project.  Despite being "12 volt"
systems, most automotive circuits run at close to 14 volts to charge the 12
volt battery.  However, during high-load conditions such as cold-cranking or
jumpstarting, voltages may be as low as 4.5 volts.  In severe case, voltages
may swing from -100 to +100 volts.  For these reasons, the power supply used
employs a number of protection techniques and switching regulators to provide
safe, consistent power to the remaining circuitry.

The input stage of the power supply provides over, under, and reverse voltage
protection.  A control chip from Linear Technology is used to drive two pass
FETs if the supply voltage is between 3.5 and 18 volts.  If the supply voltage
is outside this range, the FETs and turned off and the following circuitry is
protected.  Large transient-voltage-suppresion (TVS) diodes are also present to
protect against severe (+/- 100V) transients.  The output of this stage includes
large bulk capacitors to handle brown-outs.

The second portion of the power supply includes a dual buck switching regulator
to provide 3.3V and 5V at 3A apiece.  This portion of the supply must support 
the highest loads during normal operation.  The Raspberry Pi, AVR 
microcontroller, and vacuum-fluorescent display are all powered from 5V and 
require up 2A.  The USB to UART and OBD to UART converters both use 3.3V at 
close to 500mA.  A buck topology was chosen since the supply voltage will be
above 5V except during severe brown-out conditions.

The third and final poriton of the power supply uses a buck-boost switching
regulator to supply 12V at 1.5A.  This voltage is used to support several
of the OBD protocols.  While the raw voltage from the car itself could be used
for this purpose, more protection would be required.  This supply also allows
for different displays that may require 12V.  A buck-boost topology was chosen
since the input voltage can droop well below 12V, though most of the time it is
around 14V.

The use of switching regulators provides very consistent and efficient operation
at almost all supply voltages.  The 5 and 3.3 volt rails operate well with input
voltages from 5 to 18V and the 12 volt rail operates over the "safe" range, i.e.
3.5 to 18V.  With approximately 60% loads on each rail, the entire board is close
to 90% efficient.  While not a requirement of the project, a high-efficiency power
supply is advantageous due to decreased power draw and heat.  The power supply was
designed as a 4-layer PCB of which most the layers are ground to provide 
heat-sinking and shielding from high-frequency switching signals.

\section{Main Board}
The main board handles all functions not performed by the Raspberry Pi.  It will
provide an interface to the vehicle, receive data from the Raspberry Pi and 
display that data to the VFD.  External inputs are also provided for future 
development.  The main board was designed as a 4-layer PCB.

\subsection{AVR}
An Atmel AVR ATmega32U4 was chosen as the microcontroller to drive the VFD.  This
controller provides sufficient memory and processing power to handle this task. 
It is also used to handle switch inputs for changing brightness, display modes,
and other options.  The AVR also has the ability to output to a character LCD
for debugging.  It will communicate with the Raspberry Pi via UART.

\subsection{STN1110}
An STN1110 was selected as the interface chip between the Raspberry Pi and the 
vehicle.  The STN and its supporting circuitry handle all of the standard OBD
protcols and communicate via UART.  This chip is considered an improvement over
the ubiquitous ELM327 used in many USB and Bluetooth adapters.  The STN provides
more functionality than the ELM and includes an additional instruction set that
can streamline higher-level software.

\subsection{FTDI USB to UART}
To easily connect both the AVR and STN to the Raspberry Pi, an FTDI FT2232H UART 
to USB converter was included on the main board.  This chip will enumerate on the
Raspberry Pi as two virtual serial ports.  These ports are very easy to interface
with via C and provide an easy communication method.  The chip supports UART baud
rates up to 1MBaud, which is sufficiently fast for all operations in this project.

\section{Raspberry Pi}

A Raspberry Pi computer was chosen to be high-level platform for this system
because of its popularity and ease of use. The Raspberry Pi is connected to the
main board via USB, and delegates the tasks of gathering and displaying data
to the main board.

\subsection{Operating System}
The operating system chosen for the Raspberry Pi was Rasbian, which is the
most used and well-known operating system for the Raspberry Pi. Rasbian was
chosen for these reasons as well: we wanted to focus on designing the overall
system, not on wrestling with an obscure operating system. We prioritized ease
of development.

\section{Software}
The software for the system is written in Python 2.7.5. Python was chosen
because it is the language of choice for the Raspberry Pi, and allows us to
quickly and 
exibly develop the high-level system software. The version was
chosen for compatibility with the PySerial library, which the software uses to
communicate over USB.

Source code for the system (as well as source code for this report) can be
found at \url{https://github.com/Cantido/vehiclehud}.
