\chapter{Timeline and Deliverables}

% A simple and straightforward section to write, this should include the
% timeline from the proposal as well as the actual timeline used along with
% milestones met, not met, and altered.  A list of deliverables promised and
% delivered should be included.  Discussion here should run a few paragraphs
% and reflect your team's experiences with the schedule: were some parts of it
% rushed, did your original chart give too much time for one aspect, were some
% tasks crashed or done in parallel, etc.

\section{Spring 2013}

Through the spring semester of 2013, we solidified the goals of our project and
began some preliminary research to guide the major choices we would have to
make.

\subsection{Deliverables Promised}

Our proposal \cite{proposal}, released at the end of the spring semester of 2013,
laid out the requirements for our final product:

\begin{quote}
Our final product will be a system that can provide a Heads-Up Display to the
driver of most automobiles. This product will consist of an OBD plug to gather
vehicle data from the car, a system to display this information (such as a
dash-mounted projector), and a central computer.
\end{quote}

We did not deviate from this goal through the course of the project.

\section{Summer 2013}

The summer months were blocked out for development of the PSU and the research
of everything else. The power supply board was designed, manufacturer HUDs were
studied, and some exploratory code was written.

We had also marked this time to research our options with regard to
implementing GPS, and we found out this would be a lot more difficult than
we anticipated, due to the restrictions Google places on their Maps API.
We did not yet decide to scrap a GPS system entirely, but it lost priority to
other subsystems.

\section{Fall 2013}

A majority of the work was done during the fall semester of 2013. This was also
when our progress was documented with biweekly progress reports.

\subsection{September}

At the beginning of September, the power supply unit was already back from the
manufacturer. After the board was finished and tested, main board design began.
The main board was sent out for manufacturing near the end of this month.

During this period, the Raspberry Pi was set upi for development, and we
experimented with software. The decision was made to prototype in Python
at this time, and a completely functional implementation of the system was written.

\subsection{October}

We ran into our biggest roadblock during the month of October while we waited
for the main board to be sent back from the manufacturer. We took this time to
move the Python code base into C and C$++$, test other
operating systems for the Raspberry Pi, and test boot times on a higher-class
SD card, all for potential speed gains. Much of our time was also spent writing
this final report.

We took this opportunity to look back into implementing a navigation or GPS
system, and deeper investigation showed that it would not be possible to use
Google APIs for navigation. This optional goal was discarded.

\subsection{November}

Once the main PCB arrived, we began testing the system as a whole.
During this time the optical system was prototyped and tested, and we
continued to optimize the software. Near the end of the month, the system was
complete.

We were able to have a working system to demonstrate to our advisor on the 20th,
and any small problems with the system were cleaned up.

\section{Product Delivered}

At the end of November 2013, the major goals of the project were met. The system
functioned as promised and met all criteria set by us during the spring.

