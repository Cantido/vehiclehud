\chapter{Project Specifications}

% As in the previous section, these come mostly from the proposal and any
% deviations from the proposal must be explained.  

% The specifications include a lot of technical details:  sizes, weights,
% operational requirements, materials, costs, etc.  You may divide these into
% sections and put the details themselves in a well-formatted table, but you
% must also include at least a paragraph per section describing why these
% specifications were chosen and how they relate to the goals of the project.  

% You will find as you get into the writing of this that your ability to
% understand and explain the project increases. This makes your work more
% accessible and, since it can be communicated, more likely to actually have a
% technological impact.

% This section should be 1-3 pages in length.

\section{Power Supply Board}
\textsc{Total cost: \$ \todo{Calcualte cost}}

\section{Main Board}
\textsc{Total cost: \$ \todo{Calcualte cost}}

\section{Display Unit}
\textsc{Total cost: \$ \todo{Calcualte cost}}

\section{Raspberry Pi}
\textsc{Total cost: \$35} \\

The following specifications are beyond our control because the Raspberry Pi
is a finished product. However, the specifications suit our project well. The
small size and low power requirement allow the Raspberry Pi to work alongside
our other boards without much extra power or space. See
section~\ref{sec:raspi-design} on page~\pageref{sec:raspi-design} for more
discussion over our use of the Raspberry Pi. See reference~\cite{raspifaq} for
more general information on the Raspberry Pi computer.

\subsection{Physical Measurements}

\begin{tabular}{p{0.3\textwidth} p{0.5\textwidth}}
\hline
Dimensions & 85.60mm x 56mm x 21mm \\
Weight & 45g \\ \hline
\end{tabular}

\subsection{Hardware}

\begin{tabular}{p{0.3\textwidth} p{0.5\textwidth}} 
\hline
System-on-a-chip & Broadcom BCM2835 \\
CPU (in the SoC) & ARM1176JZFS \\
GPU & Videocore 4 \\ \hline
\end{tabular}

\subsection{Power Requirements}

\begin{tabular}{p{0.3\textwidth} p{0.5\textwidth}}
\hline
Standard & MicroUSB \\
Weight & 45g \\ \hline
\end{tabular}

\section{Optical System}
\textsc{Total cost: \todo{Calcualte cost}}
