\chapter{Project Objectives}

% You should explain and list the objectives from your proposal and explain
% any deviation the final objectives may have from those proposed. This section
% ought to be more than a bulleted list.  Each objective needs at least a
% paragraph explaining why it is included and how it was determined whether
% the objective was reached.    

% This section should be 1-2 pages, depending on how many objectives you have.

Beyond the straightforward goal of making our system display vehicle data, we
set goals that could set our project apart from other systems. See
chapter~\ref{chap:detailed-design} to understand how we met these objectives.

\section{Universality}

Our system only needs to interface in two places with the vehicle:

\begin{enumerate}

\item The OBD jack
\item The power supply

\end{enumerate}

Both of these interfaces are based on very solid standards. The OBD-II protocol
is well-documented and, by law, must be implemented by every vehicle made
today. An automobile's power supply is similarly standardized and
well-documented.

\section{Display Collimation}

It is a common practice in projected heads-up displays to ``collimate'' the
light. That means that the light has been focused such that the rays of light
are parallel, also called ``focused at infinity.'' This is done to avoid
causing eye strain due to the user focusing back and forth between the
environment and the display.

\section{Designing for an Automotive Environment}

Despite being ``12 volt'' systems, most automotive circuits run at close to 
14 volts to charge the 12 volt battery. However, during high-load conditions 
such as cold-cranking or jumpstarting, voltages may be as low as 4.5 volts.  
In severe cases, voltages may swing from $-100$ to $+100$ volts. For these 
reasons, the electronics in the system must be designed to withstand these severe
transient conditions. Most of this protection will be done with the power supply
unit (PSU).
