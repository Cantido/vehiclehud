\section{Executive Summary}

The team will design a heads-up display for use in automotive applications.
This device should be able to display vehicle information onto the
windshield of a vehicle. This would allow a driver to maintain eye contact
with the road while also viewing vehicle speed, engine information, and
even GPS displays.

The main goal of this project will be to develop a display that can
effectively show vehicle information onto the windshield regardless of
lighting conditions. Ideally, the display will also be universal, that is
it will work with most makes and models of cars and be able to display on
different types of windshields without modification. Depending on time and
financial constraints, the display may interface with GPS, infotainment,
and augmented reality products.

Major goals include:

\begin{description}
\item[Gather Data] We must gather information from the vehicle. This will
mostly like be done using the on-board communication options available on
most cars (either OBD-II or CAN bus). Optional data-gathering elements
include GPS chips and Wi-Fi cards.
\item[Process Data]Some sort of processing element must collect information
from the various data-gathering elements and process it for display.
\item[Display Data]Display options include laser projectors and LED/LCD
screens. A majority of the research in this section will concern how
projection systems will interact with automotive windshields and the safety
thereof.
\end{description}

A team of two will work on this project. Aaron Hall decided on the project
and requirements, and as an Electrical Engineering major his focus will be
on power and component integration. Robert Richter, a Computer Engineering
major, will develop the software for the system and assist Aaron in
integrating the components at a low software level.