\chapter{Executive Summary}

% Approximately one half-page description of the project. Include  the most
% important key points.

We have designed a heads-up display for use in automotive applications.
This device can project vehicle information onto the
windshield of a vehicle, allowing a driver to maintain eye contact
with the road while also viewing vehicle speed, engine information, and
several other data points.

The main goal of this project was to develop a display that can
effectively show vehicle information onto the windshield regardless of
lighting conditions. Ideally, the display would also be universal, being able
to work with most makes and models of cars and be able to display on
different types of windshields without modification. The device is also
easy to extend, relying on open-source software and commonly-used components.

The device has three major subunits:

\begin{description}
\item[Data Gathering] The device gathers information from the
vehicle using the vehicle's on-board diagnostic jack (OBD jack). This jack
provides a large amount of data beyond the straightforward vehicle speed and
engine RPM. The OBD system is based upon legally-obligated standards and is
available in all modern vehicles.
\item[Data Processing] The device uses a Raspberry Pi computer for high-level
control and data processing. The Raspberry Pi coordinates the display and
data gathering subunits and also provides a platform for further additions to
the system. 
\item[Data Display] The device uses a vacuum-fluorescent display (VFD) to
display information. Images from the VFD will be projected through a series of
focusing devices in order to be reflected on to the vehicle's windshield while
also properly focusing the light to reduce the driver's eye strain.
\end{description}

A team of two has worked on this project:

\begin{description}
\item[Aaron Hall] decided on the project and requirements, and as an Electrical
Engineering major his focus has been on power and component integration.
\item[Robert Richter,] a Computer Engineering
major, has developed the software and assisted Aaron in integrating the system.
\end{description}
